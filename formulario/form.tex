\documentclass[a4paper,11pt]{article}
\usepackage[latin1]{inputenc} 
\usepackage[pdftex]{hyperref}


\begin{document}

\begin{Form}[action={http://your-web-server.com/path/receiveform.cgi}]

\TextField[name=author,width=10cm,value={Eduardo de Santana Medeiros Alexandre}]{Autor}

\TextField[name=date,width=10cm,value=2016]{Ano}


\TextField[name=local,width=10cm,value=Brasil]{Local}

\TextField[name=orientador,width=10cm,value={Lauro Cesar Araujo}]{Orientador}

\TextField[name=coorientador,width=10cm,value={Equipe abnTeX}]{Coorientador}

\TextField[name=instituicao,width=10cm,value={Universidade do Brasil -- UBr}]{instituicao}

\TextField[name=curso,width=10cm,value={Faculdade de Arquitetura da Informacao}]{curso}


\TextField[name=programa,width=10cm,value={Programa de Pos-Graduacao}]{programa}


\ChoiceMenu[name=tipotrabalho,value=Monografia]{Tipo do trabalho}{Monografia,Dissertação,Tese}


\subsection{preambulo}

O preambulo é opcional, exemplo de texto do preambulo:

\quote{Modelo canônico de trabalho monográfico acadêmico em conformidade com
as normas ABNT apresentado à comunidade de usuários \LaTeX.}

\TextField[name=preambulo,width=10cm,multiline=true,value={Atualize o texto aqui}]{Texto do preambulo}





\end{Form}

\end{document}
